\documentclass{article}
\usepackage{graphicx}
\usepackage[utf8]{inputenc}
\usepackage[portuguese]{babel}
\usepackage{listings}
\usepackage{xcolor}
\usepackage{hyperref}
 
\lstdefinestyle{DOS}
{
    backgroundcolor=\color{white},
    basicstyle=\scriptsize\color{black}\small
}

\begin{document}
    \begin{center}
	    \includegraphics[width=100mm,scale=1]{logo-ufla.jpg}	 
    \end{center}
    	 \vspace{1cm}
    \begin{center}
        \huge\textbf{Arquitetura de Computadores II}
        \vspace{3cm}\\
       \Large{ Trabalho Prático - Pipeline}\\
        \vspace{2cm}
        \end{center}
       \Large{ Matheus Nogueira\\
       Roberto Gonçalves\\
       Gustavo Nunes\\}
   
        \vspace{3cm}
        \begin{center}
        \huge Universidade Federal de Lavras
        \end{center}
 
    \clearpage
	\begin{center}
	\Large\textbf{Instalação}
	\end{center}
	
	\begin{itemize}
	\item Agora vá na pipeline-drawning e digite para instalar(ubuntu):
	\begin{enumerate}
	\item nodejs
	\item npm (NodeJS package manager): baixado automaticamente com o NodeJS
	\item bower basta executar \textbf{sudo npm install bower -g}
	\end{enumerate}
	\item Caso a distribuição não seja o ubuntu, instale o nodejs, npm, bower.
	\item Agora vá até na pasta do projeto e execute:
	\begin{lstlisting}[style=DOS]
	$ npm install
	$ bower install
	$ gulp build
\end{lstlisting}
	\item Isso deverá gerar uma pasta chamada \textbf{dist} que irá conter a estrutura básica do programa
	\begin{center}
	\Large\textbf{Instalando o programa em um servidor FTP}
	\end{center}
	Para	utilizar	esse	programa	em	um	servidor	HTTP,	basta	você	copiar	a
pasta	\textbf{dist}	para	a	sua	pasta	\textbf{www}.	Por	exemplo:	Digamos	que	você
queira	criar	uma	subpasta	pipeline	em	seu	servidor	HTTP	local.	Basta
executar	tal	comando:\\

	\begin{center}
		sudo cp -r ./dist	 /var/www/pipeline
	\end{center}
	\clearpage
	\begin{center}
	\Large\textbf{Execução de Entrada}
	\end{center}
	O pipeline-drawning funciona através de uma configuração dupla: as instruções e a configuração das latências.\\
	Para fazer a entrada das instruções basta selecionar as instruções e dizer quais são os operandos, e selecionar a caixa caso queira adiantamento.\\
	Para configurar as latências, vá em configurações avançadas e edite as latências de cada instrução que fora previamente adicionada.
	
 \end{itemize}
\end{document}