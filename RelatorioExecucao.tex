\documentclass{article}
\usepackage{graphicx}
\usepackage[utf8]{inputenc}
\usepackage[portuguese]{babel}
\usepackage{listings}
\usepackage{xcolor}
\usepackage{hyperref}
 
\lstdefinestyle{DOS}
{
    backgroundcolor=\color{white},
    basicstyle=\scriptsize\color{black}\small
}

\begin{document}
    \begin{center}
	    \includegraphics[width=100mm,scale=1]{logo-ufla.jpg}	 
    \end{center}
    	 \vspace{1cm}
    \begin{center}
        \huge\textbf{Arquitetura de Computadores II}
        \vspace{3cm}\\
       \Large{ Trabalho Prático - Pipeline}\\
        \vspace{2cm}
        \end{center}
       \Large{ Matheus Nogueira\\
       Roberto Gonçalves\\
       Gustavo Nunes\\}
   
        \vspace{3cm}
        \begin{center}
        \huge Universidade Federal de Lavras
        \end{center}
 
    \clearpage
	\begin{center}
	\Large\textbf{Instalação}
	\end{center}
	
	\begin{itemize}
	\item Baixe e instale o \textbf{github}: 
	\begin{lstlisting}[style=DOS]
	 https://git-scm.com/downloads
\end{lstlisting}
	\item Após baixado, vá até uma pasta desejada e digite:
	\begin{lstlisting}[style=DOS]
	$ git clone https://github.com/mathnogueira/pipeline-drawing
\end{lstlisting}
	\item Agora vá na pipeline-drawning e digite para instalar:
	\begin{lstlisting}[style=DOS]
	$ sudo apt-get install nodejs
	$ sudo apt-get install npm
	$ sudo npm install bower -g
\end{lstlisting}
	\item Agora vá na pipeline-drawning e digite para instalar:
	\begin{lstlisting}[style=DOS]
	$ gulp build
	$ gulp pug
	$ gulp less
	$ gulp babel
\end{lstlisting}
	\item Agora vá na pasta build e clique no index.html


 \end{itemize}
\end{document}